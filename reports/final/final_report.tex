%% Casey Davis
%% Prof. Christian Murphy
%% CIS 350-001
%% May 4, 2012
%%
%% PROJECT FINAL REPORT

\documentclass[11pt]{article}
\usepackage[T1]{fontenc}
\usepackage[latin9]{inputenc}
\usepackage[colorlinks]{hyperref}
\usepackage[margin=1in]{geometry}
\usepackage{amsmath}
\usepackage{amssymb}
\usepackage{amsthm}
\usepackage{capt-of}
\usepackage{url}
\usepackage{graphicx}
\usepackage{color}
\usepackage{bbm}
\usepackage{latexsym}
\usepackage{xspace}
\usepackage[ruled,vlined]{algorithm2e}
\usepackage{fancyhdr}
\usepackage{titlesec}
\usepackage{newclude}
\usepackage{enumerate}
\usepackage{ae,aecompl}
\usepackage{alltt}

\pagestyle{fancy}

\lhead{\textbf{CIS 350: Software Engineering}}

\title{Algorithm Visualization Final Report}
\author{Casey Davis, Johnathan Mell, Di Mu, Federico Nusymowicz \\\\
\texttt{\{davisca, jmell, dimu, fnusy\}@seas.upenn.edu}}
\date{May 4, 2012}

\begin{document}

\maketitle

\section{Overview}

The Android application developed in this project is an algorithm visualization
application aimed at instructing young students interested in learning about
computer science algorithms through an interactive graphical user interface
presented as a competitive game.  Specifically, the application focuses on the
bin-packing problem (and the more specific knapsack problem), which involves
optimally placing valued items of certain weights into a finite number of bins
of limited capacity, maximizing the total value of the objects inside the bins.
The user interface includes the ability to touch and drag objects into bins,
notifying the user when the optimal solution has been found.

As we completed the final iteration of our project, we were able to make several
improvements to the program, mainly centered around the GUI.  By recalibrating
the main screen and adjusting the font sizes, we were able to optimize the
program for tablet use.  Furthermore, a high score screen was added and level
names were incorporated so users could see their progress as they completed
levels.

\section{Features}

\subsection{User stories}

Below is a list of the user stories we have completed as of the end of the final
iteration, along with the point values assigned to each.

\begin{enumerate}[1.]

    \item As a user who selected a difficulty level, I should see a paginated
    panel that displays all corresponding objects (3 points).

    \item As a user, I should be able to remove objects from bins one at a time
    using the paginator (2 points).

    \item As a user, I should be able to move objects from bin to bin (1 point).

    \item As a user, I should be confronted with tricky problems on harder
    difficulties, wherein greedy algorithms will not work (1 point).

    \item As a user, I would like a metric of how optimal my solution is
    compared to the algorithm solution (1 point).

    \item As a user, I should see a timer whenever I am solving a problem (1
    point).

    \item As a user, I should see my high score, as well as a high score screen
    (2 points).

    \item As a user, I would prefer graphically larger sizes of bins and bin
    objects available on a tablet screen, which is larger (2 points).

    \item As a user, I would prefer a cleaner user interface, particularly
    making the font size of text displayed on the bins and bin objects bigger,
    with different colors for objects, and a background that has depth (1
    point).

    \item As a user, I would pagination to be removed if there is only one page
    of objects (1 point).

\end{enumerate}

\textbf{Total points completed:  15} \\

With four people working on the project for three weeks, we should have achieved
at least a total point value of 12, if a point is defined as one person working
for one week.  Given that our point value was 15, this is more than our goal,
with a \textbf{project velocity of 3.75 points/person}.

\subsection{Unfinished user stories}

Most of our work was centered around the GUI during the last iteration, though a
few were focused on the high score screen.  Planned user stories include:

\begin{itemize}

    \item As a user, I would like to see the optimal solution displayed once I
    give up.

    \item As a user, I would like to see the objects look more 3D.

\end{itemize}

Improvements to the GUI were ongoing, and given more time, one-page pagination
and 3D objects would likely have been achieved.  The optimal solution we did not
have time to display, though it is calculated using our optimization algorithm.

\section{Known bugs, untested code, etc.}

\subsection{Bugs}

We have found two significant bugs in the code during this iteration:

\begin{itemize}

    \item Currently, if the high score is reset, sometimes an exception is
    thrown, causing the application to stop.  This will be resolved in the final
    modifications to the code.

    \item When traversing the levels in reverse order (using the 'Previous'
    button), their names are replaced instead with numbers.

\end{itemize}

\subsection{Untested code}

Although most of the non-trivial, complex code of the project includes a
complete test suite, much of the code involves output that is fully graphical,
given that the application's primary function is to provide a GUI for users to
learn about algorithms.

This part of the code includes files such as \texttt{BinPackingView.java} and
\texttt{BinPackingActivity.java}, which are used to develop the GUI for our
application.  Although many of the simple calculations used to position objects
on the screen can potentially be tested for proper values, ultimately the best
tests for the methods in such classes are simply to observe the results
displayed on the screen of the device or emulator used for testing.  As such,
we did not write test cases for this code.

We did, however, include a full test suite for the underlying infrastructure
upon which the GUI is built, including the Java classes representing bins and
objects, and the output of the optimization algorithm for the bin-packing
problem.  Tests for the paginator (which displays objects on multiple pages
when there are too many to fit on the screen) were also written.

\section{Potential changes}

\subsection{Proposed additional features}

There are some features that, given more time, we would like have seen
implemented in the application.  Ideally, we would like to expand the platform
to include other educational algorithm tools, such as a max-flow graph.  This
was something we discussed with the client, but did not plan to complete due to
our limited time frame.

Furthermore, we would like to add support for objects with continuous-valued
weights for the bin-packing aspect of the problem.  This would allow for more
complicated problems with difficult solutions, thought it would necessitate a
refactoring of our optimization algorithm.

Finally, a more immediately useful feature woould be a level-select screen
and buttons that return the user to the main menu (although the latter can
technically be accomplished with the back button on the majority of Android
devices.

\subsection{Developmental changes}

If we were to restart the project, there are some changes to be made.  For the
most part, the initial GUI we set up worked adequately, and the functionality
for the actual bin-packing was set up earlier.  It would, however, have been
beneficial to place more focus on planning the GUI earlier, so that perhaps we
could have explored other options such as selecting options and tapping the
bins, rather than using drag and drop (which drained a significant amount of
time in implementing its functionality).

\section{Final comments}

\subsection{Notes on the customer}

Circumstances necessitated that Dr. Murphy work with us on behalf of the
customer for the development of this application, the purpose of which was so to
see how effective virtual tools and games were on educating burgeoning computer
scientists.  Working with him was a pleasure, with feedback being given at each
stage, and flexible deadlines being established so as to ensure the group could
deliver a quality product.

\subsection{Copyright}

We would like to release the code for this project to the customer, through
Dr. Murphy, for non-profit use.

\subsection{Final code modifications}

Code modifications discussed for Homework 6 are ongoing, and will be sent to the
TA when completed.

\end{document}
