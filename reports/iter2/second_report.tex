%% Casey Davis
%% Prof. Christian Murphy
%% CIS 350-001
%% February 2, 2012
%%
%% HOMEWORK 2

\documentclass[11pt]{article}
\usepackage[T1]{fontenc}
\usepackage[latin9]{inputenc}
\usepackage[colorlinks]{hyperref}
\usepackage[margin=1in]{geometry}
\usepackage{amsmath}
\usepackage{amssymb}
\usepackage{amsthm}
\usepackage{capt-of}
\usepackage{url}
\usepackage{graphicx}
\usepackage{color}
\usepackage{bbm}
\usepackage{latexsym}
\usepackage{xspace}
\usepackage[ruled,vlined]{algorithm2e}
\usepackage{fancyhdr}
\usepackage{titlesec}
\usepackage{newclude}
\usepackage{enumerate}
\usepackage{ae,aecompl}
\usepackage{alltt}

\pagestyle{fancy}

\lhead{\textbf{CIS 350: Software Engineering}}

\title{Algorithm Visualization Second Iteration Report}
\author{Casey Davis, Johnathan Mell, Di Mu, Federico Nusymowicz \\\\
\texttt{\{davisca, jmell, dimu, fnusy\}@seas.upenn.edu}}
\date{April 10, 2012}

\begin{document}

\maketitle

\section{Overview}

This project is an Android application that is used as a tool for students to
interactively learn about algorithms in computer science.  Specifically, this
tool allows users to solve bin-packing and max-flow problems in the form of a
game.

In this iteration, we have added increased functionality to the basic operation
of the bin-packing problem application introduced in the first iteration.
Namely, we have implemented an algorithm that calculates the optimal solution
to the specified problems, removed the limit on the number of objects available
for each configuration of the problem, and allowed the removal of one object
at a time from bins.  The complete list of user stories we have completed can
be found in the Features section below.

\section{Features}

\subsection{User stories}

Below is a list of the user stories we have completed as of the end of this
iteration, as well as the point values we have assigned to each.

\begin{enumerate}

\item As a user, I would like a metric of how optimal my solution is compared to
the algorithm solution (1 point).
\item As a user, I should see my high score (1 point).
\item As a user, I would prefer larger size of bins and bin objects for devices
(e.g. tablets) whose screen size is larger (2 points).
\item As a user who selected a difficulty level, I should see a paginated panel
that displays all corresponding objects (3 points).
\item As a user, I should be able to remove objects from bins one at a time
using the paginator (2 points).
\item As a user, I should be able to move objects from bin to bin (1 point).
\item As a user, I should be confronted with tricky problems on harder
difficulties, wherein greedy algorithms won't work (1 point).

\end{enumerate}

\textbf{Total points completed:  11} \\

There are no user stories that we planned to complete before this iteration demo
that we have not completed.

\subsection{Project Velocity}

With four people working on this project for several weeks, and having completed
a total of 11 points, our project velocity is calculated to be 2.75, which is
slightly below the target velocity we hoped to work at.  The reason for this is
the development of test cases which carried no point value as they do not have
corresponding user stories.

\section{Testing, bugs, issues, etc.}

Some JUnit test cases have been developed for the application, but they are not
yet comprehensive.  A full test suite will be available by the end of the next
iteration.

There is also currently a bug in running the application on devices with a
smaller screen sizes (e.g., a typical Android smartphone).  The height of the
view in which the objects and bins are displayed are currently hard-coded into
the application, causing some parts of the app not to be displayed when it is
run on a smaller device.  This will be resolved in the next iteration.

Another minor bug in the application involves the display of text on the bins.
When objects are removed from the bin, and the amount inside is reset, the
text will occasionally display a very small number in scientific notation
rather than simply zero.  This should easily be resolved by rounding the
displayed numbers properly.


\end{document}
